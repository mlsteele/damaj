% Created 2014-11-06 Thu 02:35
\documentclass[11pt]{article}
\usepackage[utf8]{inputenc}
\usepackage[T1]{fontenc}
\usepackage{fixltx2e}
\usepackage{graphicx}
\usepackage{longtable}
\usepackage{float}
\usepackage{wrapfig}
\usepackage{soul}
\usepackage{textcomp}
\usepackage{marvosym}
\usepackage{wasysym}
\usepackage{latexsym}
\usepackage{amssymb}
\usepackage{hyperref}
\tolerance=1000
\providecommand{\alert}[1]{\textbf{#1}}

\title{DataFlowAnalysis}
\author{jessk}
\date{\today}
\hypersetup{
  pdfkeywords={},
  pdfsubject={},
  pdfcreator={Emacs Org-mode version 7.9.3f}}

\begin{document}

\maketitle

\setcounter{tocdepth}{3}
\tableofcontents
\vspace*{1cm}
\section{Building \& Running}
\label{sec-1}

  Steps to build and run our DAMAJ Decaf Compiler.
  
  These instructions assume you are running on Athena.
  
  Clone the repository or otherwise obtain a copy of the code.

  Run \verb~add -f scala~. use \verb~scala -version~ to make sure that scala is using version 2.11.2.
  We use \verb~fsc~ to compile scala a little faster. If you happen to have any trouble because of weird
  fsc version conflicts, please try killing your fsc server (find it with \verb~ps aux | grep fsc~).
  Don't set any env variables like \verb~SCALA_HOME~ as this could force you to use the wrong version
  of scala.
  
  Once you have the right version of scala run \verb~make~ or \verb~build.sh~ to compile the project.
  If you have trouble with this or later steps, try `make clean` to reset the build files.
  
  Now you should be able to run the compiler using run.sh.

  To enable optimizations you can add \verb~-O all~.

  For example:

  \verb~./run.sh --debug tests/codegen/input/01-callout.dcf -O all~

  \verb~./run.sh --debug tests/codegen/input/04-math2.dcf -o 04-math2.asm~

  Alternatively, you can use \verb~compile.sh~ to compile and run a program.
  \verb~./compile.sh tests/codegen/input/01-callout.dcf~
  This will assemble the program into \verb~tmp/out.S~, print the assembly, create an executable at \verb~tmp/out~,
  and run the executable, printing its output as well.
\section{Dataflow Analyses}
\label{sec-2}

  Relevant files: \verb~src/scala/dataflow/*~
  
  There is a trait \verb~Analysis~ from which all analyses inherit. It uses the worklist algorithm.
  Every analysis runs a transfer function on each block in a CFG. At this point in the analysis,
  every block contains at most one statement. Later, we will have analyses that run after condensing
  basic blocks in the CFG.

  Each type of analysis outputs an analysis result, parameterized by the specific type used for the analysis.
\subsection{Available Expressions}
\label{sec-2-1}

  Used in: Common Subexpression Elimination

  Operates on: Set of Expressions

  Available Expressions is a forward-running algorithm that determines whether each expression is available
  after each block. For every expression it sees, it checks whether the expression is pure. Right now, 
  every method call is declared not pure, but later we may have a finer-grained evaluation of method purity.

  If the expression is not pure, it is not a candidate for availability, its value is not cached, and all
  global variables are declared unavailable after the expression.

  If the expression is pure, then it becomes available.

  Whenever there is an assignment, the target of the assignment becomes unavailable, as do all expressions
  that use it.
\subsection{Live Variables}
\label{sec-2-2}

  Used in: Dead Code Elimination

  Operates on: Set of Variables.

  Live Variables Analysis is a backward-running algorithm that determines whether a variable will be used
  at each point in the program. 

  In every expression, every variable used in the expression is considered alive.

  The conditions of CFG forks are special-cased because otherwise the block that sets up the condition's
  temp variable will end up deleting it right before it is evaluated. This is because the condition of forks
  are not actually part of any block.

  In assignment statements, the destination of the assignment is no longer considered live.
\subsection{Reachable}
\label{sec-2-3}

  Used in: Unreachable Code Elimination

  Operates on: Booleans

  Reachable is a forward-running algorithm that consideres a block unreachable if the output of the previous
  block was unreachable. A block's output becomes unreachable if it contains a return statement.
\subsection{Reaching Definitions}
\label{sec-2-4}

  Unused as of yet. Will be used for constant propagation.

  Operates on: Set of Assignment statements.

  Reaching definitions is a forward-running algorithm which determines which assignments are still used
  at a given point.

  For each assignment statement, previous assignments to the same variable are removed from the set,
  then the current assignment is added to the set.
  
\section{Optimizations}
\label{sec-3}

  When all optimizations are enabled, they run in the order presented below.
\subsection{Common Subexpression Elimination}
\label{sec-3-1}

  
\subsection{Dead Code Elimination}
\label{sec-3-2}

  If there is an assignment to a dead variable, it is first checked for whether it is a method call. 
  If it is, it is converted simply to a method call without an assignment. Otherwise, the assignment
  to a dead variable is removed completely.

  Here is an example of before and after Dead Code Elimination runs:

  Before: 

  \includegraphics[width=0.4\textwidth]{./before_deadcode_example.png}
  
  After dead code elimination:

  \includegraphics[width=0.4\textwidth]{./after_deadcode_example.png}
\section{CFG visualization}
\label{sec-4}

    To aid in debugging, we implemented a grapher.scala that takes in a CFG,
    and auto generates a string that represents a graph which graphviz can read, and generate an svg of the graph.
    We can set a debug flag that uses grapher.scala to take the cfg in IR2, and output the string into a .gv file which is later run through dot to make an
    image of the CFG.

\end{document}
