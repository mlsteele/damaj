% Created 2014-11-06 Thu 02:26
\documentclass[11pt]{article}
\usepackage[utf8]{inputenc}
\usepackage[T1]{fontenc}
\usepackage{fixltx2e}
\usepackage{graphicx}
\usepackage{longtable}
\usepackage{float}
\usepackage{wrapfig}
\usepackage{soul}
\usepackage{textcomp}
\usepackage{marvosym}
\usepackage{wasysym}
\usepackage{latexsym}
\usepackage{amssymb}
\usepackage{hyperref}
\tolerance=1000
\providecommand{\alert}[1]{\textbf{#1}}

\title{Lab2}
\author{jessk}
\date{\today}
\hypersetup{
  pdfkeywords={},
  pdfsubject={},
  pdfcreator={Emacs Org-mode version 7.9.3f}}

\begin{document}

\maketitle

\setcounter{tocdepth}{3}
\tableofcontents
\vspace*{1cm}
\section{Building \& Running}
\label{sec-1}

  Steps to build and run our DAMAJ Decaf Compiler.
  
  These instructions assume you are running on Athena.
  
  Clone the repository or otherwise obtain a copy of the code.

  Run \verb~add -f scala~. use \verb~scala -version~ to make sure that scala is using version 2.11.2.
  We use \verb~fsc~ to compile scala a little faster. If you happen to have any trouble because of weird
  fsc version conflicts, please try killing your fsc server (find it with \verb~ps aux | grep fsc~).
  Don't set any env variables like \verb~SCALA_HOME~ as this could force you to use the wrong version
  of scala.
  
  Once you have the right version of scala run \verb~make~ or \verb~build.sh~ to compile the project.
  If you have trouble with this or later steps, try `make clean` to reset the build files.
  
  Now you should be able to run the compiler using run.sh.
  For example:
  \verb~./run.sh -t inter --debug tests/semantics/illegal/illegal-01.dcf~
  \verb~./run.sh -t inter --debug tests/semantics/legal/legal-01.dcf~
  
\section{Data Structures}
\label{sec-2}
\subsection{Symbols}
\label{sec-2-1}

   Relevant files: \verb~scala/compile/SymbolTable.scala~
   Symbol is a base class that represents any kind of object that can be stored in a symbol table.
   All symbols have an \verb~id: ID~ member, which is the name of the symbol.
\subsubsection{FieldSymbol}
\label{sec-2-1-1}

    A field symbol represents a variable (a non-method and non-callout).
    Fields have the following members:
\begin{itemize}
\item \verb~dType:DType~: The type of the variable if its scalar, or the type of each element, if the field is an array
\item \verb~size:Option[Long]~: If the variable is a scalar, this is None. Otherwise it is the length of the array
\end{itemize}
\subsubsection{CalloutSymbol}
\label{sec-2-1-2}

    Callouts only store their id.
\subsubsection{MethodSymbol}
\label{sec-2-1-3}

    This represents a method declaration
    Methods have the following members:
\begin{itemize}
\item \verb~params: SymbolTable~: The symbol table that any nested scopes should use as their parent. This symbol table contains the method's arguments.
\item \verb~returns:DType~: The return type of the method.
\end{itemize}
\subsection{SymbolTable}
\label{sec-2-2}

   Relevant files: \verb~scala/compile/SymbolTable.scala~
   \verb~SymbolTable~ represent scopes.
   Symbol tables store two things: a list of all of the variables in their scope,
   and a parent the parent scope.
   Symbol tables support the following operations:
\begin{itemize}
\item \verb~addSymbol(symbol: Symbol) : Option[Conflict]~ Attempts to add a symbol to the table. If another symbol with the same name exists in the current scope, a \verb~Conflict~ will be returned. The \verb~Conflict~ object keeps track of the first symbol found, and the duplicate second symbol.
\item \verb~addSymbol(symbols: List[Symbol]) : List[Conflict]~ Calls \verb~addSymbol~ on each symbol, returning a list of all the conflicts encountered.
\item \verb~lookupSymbol(id: ID) : Option[Symbol]~ Attempts to find a symbol by name in the current scope. This method will also attempt to look in its ancestor's scopes if the symbol was not found in the local scope.
\end{itemize}
\subsection{Intermediate Representation}
\label{sec-2-3}

   Relevant files: \verb~scala/compile/IR.scala~
   The semantic checker phase of the compiler converts the parse tree into an IR.
   The IR has the following types:
\begin{itemize}
\item \verb~ProgramIR(symbols: SymbolTable)~ Represents a valid DECAF program.
\item \verb~Block(stmst: List[Statement], fields: SymbolTable)~ A list of statements, and the symbol table associated with that scope
\end{itemize}
\subsubsection{Statement Types}
\label{sec-2-3-1}

    These types represent all of the possible DECAF statements
\begin{itemize}
\item \verb~Assignment(left: Store, right: Expr)~
\item \verb~MethodCall(method: MethodSymbol, args: List[Either[StrLiteral, Expr]])~
\item \verb~CalloutCall(callout: CalloutSymbol, args: List[Either[StrLiteral, Expr]])~
\item \verb~If(condition: Expr, thenb: Block, elseb: Option[Block])~
\item \verb~For(id: ID, start: Expr, iter: Expr, thenb: Block)~
\item \verb~While(condition: Expr, block: Block, max: Option[Long])~
\item \verb~Return(expr: Option[Expr])~
\item \verb~Break~
\item \verb~Continue~
\end{itemize}
   
\subsubsection{Expression Types}
\label{sec-2-3-2}

    These types represent all of the possible DECAF expressions
\begin{itemize}
\item \verb~BinOp(left: Expr, op: String, right: Expr)~
\item \verb~UnaryOp(op: String, right: Expr)~
\item \verb~Ternary(condition: Expr, left: Expr, right: Expr)~
\end{itemize}
\begin{itemize}

\item Load Types\\
\label{sec-2-3-2-1}%
Loads are a sub-type of expression.
\begin{itemize}
\item \verb~LoadField(from: FieldSymbol, index: Option[Expr])~ Represents loading a value from a variable or array location
\item \verb~LoadLiteral(inner: CommonLiteral)~ Represents a constant
\item \verb~Store(to: FieldSymbol, index: Option[Expr])~ Represents saving to a variable or array location
\end{itemize}

\end{itemize} % ends low level
\section{Semantic Checking}
\label{sec-3}

  Relevant files: \verb~scala/compile/ASTToIR.scala~
  The class \verb~IRBuilder~ takes an AST as an argument, and walks down the tree.
  As it traverses the tree, it calls a variety of \verb~convert~ methods, such as \verb~convertAssignment~, which convert AST types into their corresponding IR types, while doing semantic checks.
  Any of these methods can fail, which will cause add an error to a shared error list, and will propogate its failure up the tree.

\end{document}
